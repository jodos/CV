\documentclass[11pt,a4paper]{moderncv}
\moderncvtheme[blue]{classic}
\usepackage[utf8]{inputenc}
\usepackage[scale=0.92]{geometry}
\usepackage{enumitem}
\usepackage[francais]{babel}
%\usepackage{mltex}
%\usepackage{libertine}
%\usepackage{charter}
%\usepackage{newcent}
\usepackage{fourier}
%\usepackage{times}
%\usepackage{MinionPro}
%\usepackage{libertine}
%\usepackage{palatino}
%\usepackage{kurier}
%\usepackage[default,scale=0.95]{opensans}
%\usepackage[sfdefault,lf]{carlito}
%\usepackage{caladea}
%\usepackage[sfdefault]{roboto}
%\usepackage[adobe-utopia]{mathdesign}
%\usepackage[sfdefault]{AlegreyaSans}
%\usepackage{DejaVuSansMono}
%\usepackage[sfdefault]{overlock}
%\usepackage{lmodern}
%\renewcommand*\familydefault{\sfdefault}
\usepackage[T1]{fontenc}
\usepackage{tcolorbox}
\usepackage{lastpage}


\cfoot{\textit{\small{\thepage/\pageref{LastPage}}}}

\setlength{\hintscolumnwidth}{3.6cm}
\AtBeginDocument{\recomputelengths}

\firstname{Joao}
\familyname{dos Santos}
\title{Développeur Java}
\address{23 décembre 1983 \\ Célibataire \\ Suisse/Portugais \\ Permis de conduire, véhiculé \\ ~ \\ Chemin de Rionza 17 \\ 1020 Renens\\~\\}{}
\mobile{+41 76 545 37 94}
\email{joao.dossantos@outlook.com}

\extrainfo{~\\Disponibilité: de suite}
%\photo[80pt]{logo}
%\nopagenumbers{}
\pagenumbering{arabic} 


\begin{document}
\maketitle

\section{COMPÉTENCES INFORMATIQUES}
\cvcomputer{Langages de programmation}{Java, Scala, PHP}{API Java EE}{Servlet, JSF, EL, JSP, JSTL, Bean Validation, JPA, JavaMail, JAX-RS, JAX-WS, JAXB, JDBC}
\cvcomputer{EDI}{Eclipse, NetBeans, IntelliJ IDEA}{Solutions open source}{Jahia, Drupal, Alfresco, Liferay}
\cvcomputer{Gestion de projets logiciels}{Maven, Ant}{Gestionnaires de versions}{GIT, SVN, CVS}
\cvcomputer{Systèmes de suivi d'anomalies}{Mantis, Trackplus, Jira, Redmine}{Serveurs d'applications}{WebSphere, Tomcat, Jetty, Weblogic, WildFly}
\cvcomputer{Bases de données/Annuaires}{MySQL, PostgreSQL, Oracle SQL, Microsoft SQL/OpenLDAP, OpenDS, Sun LDAP Directory Server}{Programmation Web}{HTML, CSS, XML, JSON, XSLT, XSD, REST, SOAP, JavaScript (jQuery), Ajax, AngularJS, Kendo UI, Bootstrap}
\cvcomputer{Librairies Java}{Lucene, Tika, Spring LDAP, XStream, Apache Commons, iText, Log4j, Apache POI}{Frameworks}{Spring, Spring Boot, Spring MVC, Spring Data, Spring Security, Hibernate, Swing, JHipster}
\cvcomputer{Logiciels mathématiques}{Mathematica, Matlab, Maple, R}{Bureautique}{OpenOffice, Word, Excel, PowerPoint, \LaTeX}
\cvcomputer{Systèmes d'exploitation}{Linux (Debian like), Solaris, Windows}{Librairies de tests}{JUnit, Mockito, DbUnit}

\section{EXPÉRIENCE PROFESSIONNELLE}

\begin{tcolorbox}[boxrule=0pt,arc=0pt,colback=lightgray]{\textsc{Consultant IT}, \emph{QIM Info Lausanne Sàrl}, Lausanne \hfill 2019--2019}
\end{tcolorbox}
\begin{itemize}

\item[] \textbf{Tribunal Fédéral, Lausanne}

	\item[$\bullet$] \textsc{eDossier}
	
	\item[] Développement de nouvelles fonctionnalités

	\item[] \emph{\textbf{Environnement technique:}} \emph{Linux, Java, Eclipse RCP 4}
	      	      	
\end{itemize}

%\cventry{2018--2019}{\textsc{Informaticien de gestion}}{Equinoxe MIS Development Sàrl}{Lausanne}{}{}
\begin{tcolorbox}[boxrule=0pt,arc=0pt,colback=lightgray]{\textsc{Informaticien de gestion}, \emph{Equinoxe MIS Development Sàrl}, Lausanne \hfill 2018--2019}
\end{tcolorbox}
\begin{itemize}

	\item[$\bullet$] \textsc{IS-Academia}
	
	\item[] Développement de nouvelles fonctionnalités

	\item[] \emph{\textbf{Environnement technique:}} \emph{Windows, PL/SQL, XSLT, XML}
	      	      	
\end{itemize}

\vspace{2cm}



%\cventry{2017--2018}{\textsc{Développeur Java Web}}{Academic Work Switzerland S.A.}{Genève}{}{}
\begin{tcolorbox}[boxrule=0pt,arc=0pt,colback=lightgray]{\textsc{Développeur Java Web}, \emph{Academic Work Switzerland S.A.}, Genève \hfill 2017--2018}
\end{tcolorbox}
\begin{itemize}

\item[] \textbf{Scheuchzer, Bussigny}

	\item[$\bullet$] \textsc{Application Web liée à la LTr}
	\item[] Reprise du projet et analyse de l'application $\cdot$ Gestion et automatisation de production de l'application avec Maven $\cdot$ Optimisations de l'interface Web $\cdot$ Amélioration des temps de réponse, mise en place d'un cache $\cdot$ Statistiques sur l'utilisation des machines par étape et type d'activité, requête SQL sur plusieurs tables $\cdot$ Personne de contact pour les aspects techniques et métiers

\item[] \emph{\textbf{Environnement technique:}} \emph{Windows, Microsoft SQL Server 2012, Microsoft SQL Server Management Studio, Java 6, SVN, NetBeans, Hibernate 3/JPA 2, JAXB, XSD, XML, Joda-Time, Apache POI, Google Distance Matrix API, HTML, Javascript, BootStrap, CSS, Moment.js, Maven, REST, Jersey, Quartz, Ehcache, JRebel, Tomcat 6}
	      	      	
\end{itemize}


%\cventry{2015--2016}{\textsc{Développeur Java Web}}{SQLI (Suisse) S.A.}{Renens}{}{}
\begin{tcolorbox}[boxrule=0pt,arc=0pt,colback=lightgray]{\textsc{Développeur Java Web}, \emph{SQLI (Suisse) S.A.}, Renens \hfill 2015--2016}
\end{tcolorbox}
\begin{itemize}

	\item[] \textbf{DSI Canton de Vaud --- Pôle fiscalité, Renens}
	
	\item[$\bullet$] \textsc{Refonte de l’intranet pour la taxation des PM du Canton de Vaud selon la méthodologie agile (SCRUM) en intégration continue}
	\item[] Découpage de tâches en sous-tâches $\cdot$ Estimation en "Story Points" des sous-tâches $\cdot$ Développement de flux Editique (Décisions de taxation) $\cdot$ Développement de messages SIPF (Amendes) $\cdot$ Couverture des services et des DAO par des tests unitaires $\cdot$ Conception et développement d'une fenêtre personnalisée pour la création des notes de taxation avec Kendo UI $\cdot$ Développement règles de calcul (Report des pertes)
	\item[] \emph{\textbf{Environnement technique:}} \emph{Windows, Oracle 12c, SQL Developer, Java 8, GIT, IntelliJ IDEA, Spring 4, Spring Boot, Spring MVC, Spring Data, Hibernate 4/JPA 2, JAXB, XSD, XML, Xpath, AngularJS, HTML5, Javascript, Kendo UI, BootStrap, CSS3, JUnit 4, Mockito, Maven, REST, Jersey, Tomcat 8}
\end{itemize}



%\cventry{2013--2015}{\textsc{Analyste-développeur Java/Java EE}}{MIT Micro informatique et technologies S.A.}{Bussigny}{}{}
\begin{tcolorbox}[boxrule=0pt,arc=0pt,colback=lightgray]{\textsc{Analyste-développeur Java/Java EE}, \emph{MIT Micro informatique et technologies S.A.}, Bussigny \hfill 2013--2015}
\end{tcolorbox}
\begin{itemize}
		
	\item[$\bullet$] \textsc{TRAC (Trade Risk Active Control)}
	\item[] Développement de nouvelles fonctionnalités $\cdot$ Masquage des données sensibles en fonction d'un rôle lors de l'affichage au client
	\item[] \emph{\textbf{Environnement technique:}} \emph{Windows, PL/SQL, Oracle 11g, Java/Java EE, Weblogic, WebSphere, Oracle ADF Faces, SQL Developer, JDeveloper, Oracle Data Redaction}

	\item[$\bullet$] \textsc{Interfaces métiers}
	\item[] Conception et développement d'interfaces (comptes, commodités, acteurs, taux de change) entre TRAC et les systèmes internes d'une banque (architecture web services SOAP) $\cdot$ Test fonctionel avec SoapUI $\cdot$ Intégration de ces interfaces à TRAC $\cdot$  Support technique et fonctionelle
	\item[] \emph{\textbf{Environnement technique:}} \emph{Linux (CentOS), PL/SQL, Oracle 11g, Java/Java EE, Weblogic, WebSphere, SQL Developer, SoapUI, Eclipse, SOAP, XML, XSD, JAXB, Apache CXF, JAX-WS, WSDL, wsimport, xjc, XMLBeans, Maven, SVN}	      	      	
\end{itemize}



%\cventry{2012--2013}{\textsc{Développeur Java/Java EE}}{Lab4Tech: Entreprise de formation}{Lausanne}{}{}
\begin{tcolorbox}[boxrule=0pt,arc=0pt,colback=lightgray]{\textsc{Développeur Java/Java EE}, \emph{Lab4Tech: Entreprise de formation}, Lausanne \hfill 2012--2013}
\end{tcolorbox}

\begin{itemize}
		
	\item[$\bullet$] \textsc{JSP News --- Gestion d'abonnements à des news}    	
	\item[] Création du schéma de la base $\cdot$ Formulaire de connexion $\cdot$ Gestion des procédures pour s'abonner et se désabonner
	\item[] \emph{\textbf{Environnement technique:}} \emph{Windows, Java/Java EE, MySQL, Jetty, JSP}
	      	      	
	\item[$\bullet$] \textsc{VideoSearch --- Recherche multicritère sur une BD} \url{https://github.com/jodos/videosearch}
	\item[] Création de l'interface graphique $\cdot$ Création des requêtes SQL $\cdot$ Création de la partie automatique     	
	\item[] \emph{\textbf{Environnement technique:} {Windows, Java, MySQL, Swing}}
	      	      	
\end{itemize}

\vspace{2cm}

%\cventry{2010--2012}{\textsc{Ingénieur étude et développements}}{Smile Suisse: 1\up{er} intégrateur européen de solutions Open Source}{Genève}{}{}
\begin{tcolorbox}[boxrule=0pt,arc=0pt,colback=lightgray]{\textsc{Ingénieur étude et développements}, \emph{Smile Suisse S.A.}, Genève \hfill 2010--2012}
\end{tcolorbox}

\begin{itemize}
		
	\item[] \textbf{Tribunal Fédéral, Lausanne}
	
	\item[$\bullet$] \textsc{OpenJustitia---outil de recherche des décisions du tribunal basé sur Alfresco}
	\item[] Intégration d'un nouveau type de document $\cdot$ Modifications de l'interface web $\cdot$ Développement d'un module de complètement automatique à partir de l'index Lucene et d'un thésaurus $\cdot$ Rédaction d'un document technique en anglais sur l'architecture d'OpenJustitia 	
	\item[] \emph{\textbf{Environnement technique:} Solaris, PostgreSQL, Java/Java EE, Alfresco 3-4, Tomcat 6, JSF, Hibernate, Spring, Lucene, CVS, Ant}
	      	      	      	
	\item[$\bullet$] \textsc{Intranet documentaire avec Jahia et Alfresco}
	\item[] Développement des templates $\cdot$ Développement d'un module de recherche sur la base d'un annuaire LDAP $\cdot$ Mise en place du connecteur entre Jahia et Alfresco $\cdot$ Mise en place d'un workflow au sein du Tribunal Fédéral $\cdot$ Formations contributeur et administrateur $\cdot$ Support technique de niveau 2 $\cdot$ Suivi des anomalies en anglais avec l'équipe d'experts techniques de Jahia $\cdot$ Mises en production des corrections et des évolutions
	\item[] \emph{\textbf{Environnement technique:} Solaris, PostgreSQL, Java/Java EE, Jahia 6, Tomcat 6, JSP, JSTL, Jahia taglibs, Spring LDAP, Hibernate, Spring, Tika, Sun LDAP Directory Server, Trackplus, SVN, Maven, Jira}
	      	      	      	
	\item[] \textbf{UBP, Genève}
	
	\item[$\bullet$] \textsc{Intranet avec Jahia}
	\item[] Développement des templates $\cdot$ Développement d'un module de recherche sur la base d'un annuaire LDAP
	\item[] \emph{\textbf{Environnement technique:} Linux (Ubuntu), MySQL, Java/Java EE, Jahia 6, Tomcat 6, 	 JSP, JSTL, Jahia taglibs, JNDI, Hibernate, Spring, jQuery, Ajax, OpenLDAP, SVN, Ant, Redmine, Mantis}
	      	      	      	
	\item[] \textbf{HUG, Genève}
	
	\item[$\bullet$] \textsc{Intranet avec Drupal}
	\item[] Installation de Drupal en mode usine à sites $\cdot$ Développement des templates $\cdot$ Développement de modules $\cdot$ Paramétrage des différents modules $\cdot$ Rédaction d'un manuel utilisateur
	\item[] \emph{\textbf{Environnement technique:} Linux (Ubuntu), MySQL, PHP, Drupal 6, Apache, jQuery, CSS, SVN, Mantis}
	      	      	      	
	%\item[] \textbf{Mission Permanente du Pérou}
	
	%\item[$\bullet$] \textsc{Mise en place d'Alfresco pour la gestion de leur flux de documents (20 jours)}
	%\item[] Installation d'Alfresco $\cdot$ Intégration d'Alfresco avec un annuaire LDAP $\cdot$ Récupération de mails et stockage sur Alfresco $\cdot$ Gestion des droits sur les différents dossiers $\cdot$ Mise en place d'un workflow simple	
	%\item[] Environnement technique: Alfresco 3, OpenLDAP
	      	      	
\end{itemize}

\section{CERTIFICATIONS}
%\cventry{en cours}{Python for Everybody}{Université du Michigan/Coursera.}{}{}{}
\cventry{septembre 2018}{Version Control with Git}{Atlassian/Coursera}{}{}{}
\cventry{août 2018}{MATLAB Onramp}{MathWorks}{}{}{}
\cventry{novembre 2016}{Professional Scrum Developer I}{Scrum.org}{}{}{}
\cventry{octobre 2016}{Professional Scrum Master I}{Scrum.org}{}{}{}
\cventry{mars 2013}{Scrum par la pratique}{Pyxis}{}{}{}
\cventry{janvier 2013}{Oracle Certified Professional, Java SE 6 Programmer}{Oracle}{}{}{}

\section{FORMATION}
\cventry{2007--2009}{Bachelor of Science HES-SO en Télécommunications}{HEIG-VD}{Yverdon-les-Bains}{cours à option: Recherche d'informations multimédias, Systèmes bio-inspirés, Technologie XML, Programmation WEB}{Projet de bachelor: Système bio-inspiré pour l'analyse de données biomédicales, application au diagnostic du cancer. \url{http://tb.heig-vd.ch/2841}}
\cventry{2003--2007}{Études en informatique}{EPFL}{Lausanne}{}{}
\cventry{1999--2003}{Maturité fédérale}{Gymnase de Beaulieu}{Lausanne}{option physique et applications des mathématiques}{Travail de maturité: Factorisation des nombres entiers.}


\section{LANGUES}
\cvcomputer{Portugais}{Langue maternelle}{Anglais}{Niveau intermédiaire, B1}
\cvcomputer{Français}{Langue maternelle}{}{}

\end{document} 
