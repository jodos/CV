\documentclass[10pt,a4paper]{moderncv}
\moderncvtheme[blue]{classic}
\usepackage[utf8]{inputenc}
\usepackage[scale=0.9]{geometry}
\usepackage{enumitem}
\usepackage[utf8]{inputenc}
\usepackage[francais]{babel}
%\usepackage{mltex}
%\usepackage{libertine}
%\usepackage{charter}
%\usepackage{newcent}
\usepackage{fourier}
%\usepackage{times}
%\usepackage{MinionPro}
%\usepackage{libertine}
%\usepackage{palatino}
%\usepackage{kurier}

\setlength{\hintscolumnwidth}{3.6cm}
\AtBeginDocument{\recomputelengths}

\firstname{Joao}
\familyname{dos Santos}
%\title{Développeur Java}
\address{33 ans \\ Célibataire \\ Suisse/Portugais\\~\\ Chemin de Rionza 17 \\ 1020 Renens\\~\\}{}
\mobile{+41 76 545 37 94}
\email{joao.dossantos@outlook.com}

\extrainfo{~\\Disponibilité: à convenir}
%\photo[80pt]{logo}
\nopagenumbers{}


\begin{document}
\maketitle

\section{COMPÉTENCES INFORMATIQUES}
\cvcomputer{Systèmes d'exploitation}{Linux (Debian like), Solaris, Windows}{Langages de programmation}{Java (Jave EE, Java SE, Java ME), Ada, Scala, C, Perl, PHP}
\cvcomputer{EDI}{Eclipse, NetBeans, IntelliJ IDEA}{Solutions open source}{Jahia, Drupal, Alfresco, Liferay}
\cvcomputer{Gestion de projets logiciels}{Maven, Ant}{Gestionnaires de versions}{GIT, SVN, CVS}
\cvcomputer{Systèmes de suivi d'anomalies}{Mantis, Trackplus, Jira, Redmine}{Bureautique}{OpenOffice, Word, Excel, PowerPoint, \LaTeX}
\cvcomputer{Bases de données/Annuaires}{MySQL, PostgreSQL, Oracle SQL, Microsoft SQL/OpenLDAP, OpenDS, Sun LDAP Directory Server}{Programmation Web}{HTML, CSS, XML, JSON, XSLT, XSD, REST, SOAP, JavaScript (jQuery), Ajax, AngularJS, Kendo UI, Bootstrap}
\cvcomputer{Librairies Java}{Lucene, Tika, Spring LDAP, XStream, Apache Commons, iText, Log4j, Apache POI}{Frameworks}{Spring 4, Spring Boot, Spring MVC, Spring Data, Spring Security, Hibernate, Swing}
\cvcomputer{Logiciels mathématiques}{Mathematica, Matlab, Maple, R}{Serveurs d'applications}{WebSphere, Tomcat, Jetty, Weblogic, WildFly}
\cvcomputer{API Java EE}{Servlet, JSF, EL, JSP, JSTL, Bean Validation, JPA, JavaMail, JAX-RS, JAX-WS, JAXB, JDBC}{Librairies de tests}{JUnit, Mockito, DbUnit}

\section{EXPÉRIENCE PROFESSIONNELLE}

\cventry{2017--en cours}{\textsc{Développeur Java Web}}{Scheuchzer S.A. [mission]}{Bussigny}{}{}
\begin{itemize}

	\item[$\bullet$] \textsc{Application Web liée à la LTr}
	\item[] Migration du projet sur Maven $\cdot$ Optimisations de l'interface Web $\cdot$ Amélioration des temps de réponse $\cdot$ Statistiques (SQL) sur l'utilisation des machines

	\item[] \emph{\textbf{Environnement technique:}} Windows, Microsoft SQL Server 2012, Microsoft SQL Server Management Studio, Java 6, SVN, NetBeans, Hibernate 3/JPA 2, JAXB, XSD, XML, Joda-Time, Apache POI, Google Distance Matrix API, HTML, Javascript, BootStrap, CSS, Moment.js, Maven, REST, Jersey, Quartz, Ehcache, JRebel, Tomcat 6
	      	      	
\end{itemize}

\vspace{1cm}

\cventry{2015--2016}{\textsc{Développeur Java Web}}{SQLI (Suisse) S.A. [CDI]}{Renens}{}{}
\begin{itemize}

	\item[] \textbf{DSI Canton de Vaud --- Pôle fiscalité}
	
	\item[$\bullet$] \textsc{Refonte de l’intranet pour la taxation des PM du Canton de Vaud selon la méthodologie agile (SCRUM) en intégration continue}
	\item[] Découpage de tâches en sous-tâches $\cdot$ Estimation en "Story Points" des sous-tâches $\cdot$ Développement de flux Editique (Décisions de taxation) $\cdot$ Développement de messages SIPF (Amendes) $\cdot$ Couverture des services et des DAO par des tests unitaires $\cdot$ Conception et développement d'une fenêtre personnalisée pour la création des notes de taxation avec Kendo UI $\cdot$ Développement règles de calcul (Report des pertes)
	\item[] \emph{\textbf{Environnement technique:}} Windows, Oracle 12c, SQL Developer, Java 8, GIT, IntelliJ IDEA, Spring 4, Spring Boot, Spring MVC, Spring Data, Hibernate 4/JPA 2, JAXB, XSD, XML, Xpath, AngularJS, HTML5, Javascript, Kendo UI, BootStrap, CSS3, JUnit 4, Mockito, Maven, REST, Jersey, Tomcat 8	
\end{itemize}


\clearpage

\cventry{2013--2015}{\textsc{Analyste-développeur Java/Java EE}}{MIT Micro informatique et technologies S.A. [CDI]}{Bussigny}{}{}
\begin{itemize}
		
	\item[$\bullet$] \textsc{TRAC (Trade Risk Active Control)}
	\item[] Développement de plusieurs écrans et ajout de nouvelles fonctionnalités		
	\item[] \emph{\textbf{Environnement technique:}} Windows, PL/SQL, Oracle 11g, Java/Java EE, Weblogic, WebSphere, Oracle ADF Faces, SQL Developer, JDeveloper

	\item[$\bullet$] \textsc{Interfaces métiers}
	\item[] Conception et développement d'interfaces (comptes, commodités, acteurs, taux de change) entre TRAC et les systèmes internes d'une banque $\cdot$ Support technique et fonctionelle      	 	
	\item[] \emph{\textbf{Environnement technique:}} Linux (CentOS), PL/SQL, Oracle 11g, Java/Java EE, Weblogic, WebSphere, SQL Developer, SoapUI, Eclipse, SOAP, XML, XSD, JAXB, Apache CXF, JAX-WS, WSDL, wsimport, xjc, XMLBeans, Maven, SVN	      	      	
\end{itemize}

\vspace{1cm}

\cventry{2010--2012}{\textsc{Ingénieur étude et développements}}{Smile Suisse: 1\up{er} intégrateur européen de solutions Open Source [CDI]}{Genève}{}{}
\begin{itemize}
		
	\item[] \textbf{Tribunal Fédéral}
	
	\item[$\bullet$] \textsc{OpenJustitia---outil de recherche des décisions du tribunal basé sur Alfresco (6 mois)}
	\item[] Intégration d'un nouveau type de document $\cdot$ Modifications de l'interface web $\cdot$ Développement d'un module de complètement automatique à partir de l'index Lucene et d'un thésaurus $\cdot$ Rédaction d'un document technique en anglais sur l'architecture d'OpenJustitia 	
	\item[] \emph{\textbf{Environnement technique:}} Solaris, PostgreSQL, Java/Java EE, Alfresco 3-4, Tomcat 6, JSF, Hibernate, Spring, Lucene, CVS, Ant
	      	      	      	
	\item[$\bullet$] \textsc{Intranet documentaire avec Jahia et Alfresco (11 mois)}
	\item[] Développement des templates $\cdot$ Développement d'un module de recherche sur la base d'un annuaire LDAP $\cdot$ Mise en place du connecteur entre Jahia et Alfresco $\cdot$ Mise en place d'un workflow au sein du Tribunal Fédéral $\cdot$ Formations contributeur et administrateur $\cdot$ Support technique de niveau 2 $\cdot$ Suivi des anomalies en anglais avec l'équipe d'experts techniques de Jahia $\cdot$ Mises en production des corrections et des évolutions
	\item[] \emph{\textbf{Environnement technique:}} Solaris, PostgreSQL, Java/Java EE, Jahia 6, Tomcat 6, JSP, JSTL, Jahia taglibs, Spring LDAP, Hibernate, Spring, Tika, Sun LDAP Directory Server, Trackplus, SVN, Maven, Jira
	      	      	      	
	\item[] \textbf{Union Bancaire Privée}
	
	\item[$\bullet$] \textsc{Intranet avec Jahia (4 mois)}
	\item[] Développement des templates $\cdot$ Développement d'un module de recherche sur la base d'un annuaire LDAP
	\item[] \emph{\textbf{Environnement technique:}} Linux (Ubuntu), MySQL, Java/Java EE, Jahia 6, Tomcat 6, JSP, JSTL, Jahia taglibs, JNDI, Hibernate, Spring, jQuery, Ajax, OpenLDAP, SVN, Ant, Redmine, Mantis
	      	      	      	
	\item[] \textbf{Hôpitaux Universitaires de Genève}
	
	\item[$\bullet$] \textsc{Intranet avec Drupal (4 mois)}
	\item[] Installation de Drupal en mode usine à sites $\cdot$ Développement des templates $\cdot$ Développement de modules $\cdot$ Paramétrage des différents modules $\cdot$ Rédaction d'un manuel utilisateur
	\item[] \emph{\textbf{Environnement technique:}} Linux (Ubuntu), MySQL, PHP, Drupal 6, Apache, jQuery, CSS, SVN, Mantis
	      	      	      	
	%\item[] \textbf{Mission Permanente du Pérou}
	
	%\item[$\bullet$] \textsc{Mise en place d'Alfresco pour la gestion de leur flux de documents (20 jours)}
	%\item[] Installation d'Alfresco $\cdot$ Intégration d'Alfresco avec un annuaire LDAP $\cdot$ Récupération de mails et stockage sur Alfresco $\cdot$ Gestion des droits sur les différents dossiers $\cdot$ Mise en place d'un workflow simple	
	%\item[] Environnement technique: Alfresco 3, OpenLDAP
	      	      	
\end{itemize}

\section{CERTIFICATIONS}
\cventry{novembre 2016}{Professional Scrum Developer I}{Scrum.org}{}{}{}
\cventry{octobre 2016}{Professional Scrum Master I}{Scrum.org}{}{}{}
\cventry{mars 2013}{Scrum par la pratique}{Pyxis}{}{}{}
\cventry{janvier 2013}{Oracle Certified Professional, Java SE 6 Programmer}{Oracle}{}{}{}

\section{FORMATION}
\cventry{2007--2009}{Bachelor of Science HES-SO en Télécommunications}{HEIG-VD}{Yverdon-les-Bains}{cours à option: Recherche d'informations multimédias, Systèmes bio-inspirés, Technologie XML, Programmation WEB}{Projet de bachelor: Système bio-inspiré pour l'analyse de données biomédicales, application au diagnostic du cancer. \url{http://tb.heig-vd.ch/2841}}
\cventry{2003--2007}{Études en informatique}{EPFL}{Lausanne}{}{}
\cventry{1999--2003}{Maturité fédérale}{Gymnase de Beaulieu}{Lausanne}{option physique et applications des mathématiques}{Travail de maturité: Factorisation des nombres entiers.}


\section{LANGUES}
\cvcomputer{Portugais}{Langue maternelle}{Anglais}{Niveau intermédiaire, B1}
\cvcomputer{Français}{Langue maternelle}{Allemand}{Connaissances de base, A1}

\end{document} 